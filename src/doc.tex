\documentclass[12pt, a4paper]{article}

% Пакеты для поддержки русского языка
\usepackage[T2A]{fontenc}
\usepackage[utf8]{inputenc}
\usepackage[russian]{babel}

% Дополнительные пакеты для оформления
\usepackage{fancyhdr} % Для колонтитулов
\usepackage{hyperref} % Для гиперссылок и оглавления
\usepackage{listings} % Для красивого форматирования кода
\usepackage{verbatim} % Для вывода "как есть"

% Настройка колонтитулов
\pagestyle{fancy}
\fancyhf{}
\rhead{Документация}
\lhead{Проект "Тетрис"}
\rfoot{Стр. \thepage}

% Настройки hyperref
\hypersetup{
    colorlinks=true,
    linkcolor=blue,
    filecolor=magenta,
    urlcolor=cyan,
}

\title{Документация к проекту "Тетрис"}
\date{\today}

\begin{document}

\maketitle

\tableofcontents

\newpage

\section{Введение}

Проект "Тетрис" представляет собой консольную реализацию классической игры, разработанную на языке C с использованием библиотеки \texttt{ncurses} для создания текстового псевдографического интерфейса. Целью проекта является создание функциональной и расширяемой версии игры, соответствующей современным стандартам программирования, включая модульное тестирование.

\section{Архитектура и структура кода}

Проект организован в виде нескольких модулей, каждый из которых отвечает за свою часть функционала:

\begin{itemize}
    \item \texttt{backend.h}: Этот заголовочный файл определяет основные структуры данных и функции, отвечающие за игровую логику. Здесь находятся определения игрового поля, тетрамино, а также функции для их инициализации, перемещения, вращения, проверки коллизий, подсчета очков и сохранения рекордов.
    \item \texttt{entities.h}: В этом файле содержатся ключевые перечисления и структуры, которые описывают состояние игры (\texttt{Tetris\_state}), сигналы управления (\texttt{Tetris\_signal}), действия пользователя (\texttt{UserAction\_t}) и типы тетрамино (\texttt{Piece\_type}). Это центральный файл для определения всех сущностей игры.
    \item \texttt{fsm.h}: Определяет логику конечного автомата (FSM), который управляет переходами между различными состояниями игры (например, \texttt{STATE\_START}, \texttt{STATE\_MOVING}, \texttt{STATE\_PAUSE}, \texttt{STATE\_GAMEOVER}). Этот модуль обрабатывает сигналы от пользователя и внутренней логики игры для изменения её состояния.
    \item \texttt{cli.h}: Отвечает за интерфейс командной строки. Этот модуль содержит функции для инициализации и очистки \texttt{ncurses}, отрисовки игрового поля, следующей фигуры и всей игровой информации (счет, уровень, рекорд), а также для вывода сообщений пользователю.
\end{itemize}

\section{Игровой процесс и управление}

Игра начинается в состоянии \texttt{STATE\_START}. После нажатия клавиши \texttt{Enter} игра переходит в состояние \texttt{STATE\_SPAWN}, где появляется новая фигура.

Управление в игре осуществляется следующими клавишами:
\begin{itemize}
    \item \textbf{Стрелка влево:} Переместить фигуру влево.
    \item \textbf{Стрелка вправо:} Переместить фигуру вправо.
    \item \textbf{Стрелка вниз:} Плавное ускорение падения (\textit{soft drop}).
    \item \textbf{Стрелка вверх:} Поворот фигуры.
    \item \textbf{Пробел:} Мгновенное падение фигуры (\textit{hard drop}).
    \item \textbf{P:} Пауза в игре.
    \item \textbf{Escape или Q:} Выход из игры.
\end{itemize}

\section{Системы игры}

\begin{itemize}
    \item \textbf{Счёт и уровни:} Очки начисляются за каждую очищенную линию. При достижении определенного количества очков скорость падения фигуры увеличивается, а уровень игры повышается. Максимальный уровень - 10.
    \item \textbf{Рекорды:} Игра поддерживает сохранение и загрузку рекордов, что позволяет отслеживать лучшие результаты.
    \item \textbf{Механика мешка:} Для спавна фигур используется "мешок", который гарантирует появление каждой из 7 фигур тетрамино один раз в каждом цикле, что делает игру более справедливой и менее предсказуемой, чем случайное появление.
\end{itemize}

\section{Сборка, тестирование и документация}

Проект использует \texttt{Makefile} для автоматизации основных задач:
\begin{itemize}
    \item \texttt{make all}: Компиляция проекта. Исполняемый файл \texttt{tetris} будет создан в директории \texttt{build}.
    \item \texttt{make test}: Сборка и запуск модульных тестов, определенных в модуле \texttt{tests}.
    \item \texttt{make gcov\_report}: Генерация отчета о покрытии кода тестами. Отчет сохраняется в виде HTML-файла в директории \texttt{gcov\_report}.
    \item \texttt{make dvi}: Генерация документации из исходного файла \texttt{doc.tex}.
\end{itemize}

\end{document}